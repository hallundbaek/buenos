\documentclass[a4paper,12pt]{article}

% Standard configuration
\usepackage[english]{babel}
\usepackage[utf8]{inputenc}
\usepackage[T1]{fontenc}
\usepackage{hyperref}
\usepackage{setspace}
\onehalfspacing
\usepackage{upquote}
\usepackage{amsmath,amssymb}
\usepackage{tikz}
\usepackage[margin=1.3in]{geometry}
\usepackage[numbers]{natbib}

% Listings configuration
\usepackage{color}
\definecolor{listinggray}{gray}{0.9}
\usepackage{multirow,bigdelim}
\usepackage{listings}
\lstset{
	language=,
	literate=
		{æ}{{\ae}}1
		{ø}{{\o}}1
		{å}{{\aa}}1
		{Æ}{{\AE}}1
		{Ø}{{\O}}1
		{Å}{{\AA}}1,
	backgroundcolor=\color{listinggray},
	tabsize=3,
	rulecolor=,
	basicstyle=\scriptsize,
	upquote=true,
	aboveskip={1.5\baselineskip},
	columns=fixed,
	showstringspaces=false,
	extendedchars=true,
	breaklines=true,
	prebreak =\raisebox{0ex}[0ex][0ex]{\ensuremath{\hookleftarrow}},
	frame=single,
	showtabs=false,
	showspaces=false,
	showstringspaces=false,
	identifierstyle=\ttfamily,
	keywordstyle=\color[rgb]{0,0,1},
	commentstyle=\color[rgb]{0.133,0.545,0.133},
	stringstyle=\color[rgb]{0.627,0.126,0.941},
}
%captions on listings
\usepackage[center,font=small,labelfont=bf,textfont=it]{caption}

% Header configuration
\usepackage{fancyhdr}
\lhead[]{} %clear standard settings
\chead[]{} %clear standard settings
\rhead[]{} %\rightmark} %current section
\lfoot[]{} %clear standard settings
\cfoot[]{\thepage} %current page number 
\rfoot[]{} %clear standard settings

\title{OSM\\G assignment 2}
\author{Tobias Hallundbæk Petersen (xtv657)\\Ola Rønning (vdl761)\\Nikolaj Høyer (ctl533)}
\date{February 24, 2014}

\begin{document}
\maketitle
\tableofcontents
\newpage
\section{Types and Functions for Userland Processes in Buenos}
\subsection{Define a data structure to represent a user process}

We start by defining the data structure needed for representing user processes. This is done in \texttt{syscall.h}. First we define a process state as consisting of a number of different states, then we define our process control block as using this process state, along with a process id (an integer), a parent id, a name, a return value and an integer representing the number of child processes.
\begin{lstlisting}
#define CONFIG_MAX_NAME 128
typedef enum {
  PROC_FREE,
  PROC_RUNNING,
  PROC_READY,
  PROC_SLEEPING,
  PROC_ZOMBIE,
  PROC_NONREADY,
  PROC_DYING,
  PROC_NOTFREE
} process_state_t;

typedef struct {
  process_state_t state;
  process_id_t id;
  process_id_t parentid;
  char name[CONFIG_MAX_NAME];
  int retval;
  int children;
} process_control_block_t;
\end{lstlisting}

\subsection{Implement a library of helper functions}
We start by modifying the existing \texttt{process\_start} function, making it take a process id as input, instead of an executable. We make sure that the thread entry \texttt{my\_entry} gets the process id of the input, and also we modify the \texttt{file = ...} line to update the process table with the process name, as well as opening the file. This also requires a spinlock, to prevent corruption of the process table.
\begin{lstlisting}
void process_start(process_id_t pid)
{
    ...
    my_entry->process_id = pid;
    ...
    spinlock_acquire(&process_lock);
    file = vfs_open((char *)process_table[pid].name);
    spinlock_release(&process_lock);
    ...
\end{lstlisting}

We also need a helper function, which creates a new process id. Such a function is defined below. We look through the process table and checks for a free process. This is then returned. Also, a spinlock is required when accessing the process table. If this in some way or another fails, $-1$ is returned.
\begin{lstlisting}
process_id_t process_new_id() {
  int pid;
  process_control_block_t process;
  spinlock_acquire(&process_lock);
  for (pid = 0; pid < CONFIG_MAX_PROCESSES; pid ++) {
    process = process_table[pid];
    if (process.state == PROC_FREE){
      process.parentid = -1;
      process.children = 0;
      process.state = PROC_NOTFREE;
      spinlock_release(&process_lock);
      return pid;
    }
  }
  spinlock_release(&process_lock);
  return -1;
}
\end{lstlisting}

\subsubsection{\texttt{process\_spawn}}

When spawning a process, we give an executable as input and return its process id. Thus, we need the helper function described above - we start by acquiring an available process id, then update our process table. Again, we apply a spinlock before making changes to the table. The parent process for the new process must be the current running process, hence \texttt{parentid} is set with the value of \texttt{process\_get\_current\_process()}. We also use \texttt{stringcopy} to place the name of the executable in the table. The other table values should be clear from the context.

We then make a check - if \texttt{pid} is \texttt{0}, then it is our init process running. If this is the case, we simply set the process as running and return. Otherwise, its some sub-process and we create a new thread for the child process and run it. Also, we increment the parents child-counter.

\begin{lstlisting}
process_id_t process_spawn(const char *executable) {
  process_id_t pid;
  TID_t child_tid;
  pid = process_new_id();

  spinlock_acquire(&process_lock);
  process_id_t parent_process = process_get_current_process();
  process_table[pid].state = PROC_READY;
  process_table[pid].parentid = parent_process;
  process_table[pid].id = pid;
  process_table[pid].children = 0;
  stringcopy(process_table[pid].name,executable,CONFIG_MAX_NAME);
  if (pid == 0) {
    process_table[pid].state = PROC_RUNNING;
    spinlock_release(&process_lock);
    return pid;
  }
  child_tid = thread_create((void (*)(uint32_t))process_start,(uint32_t) pid );
  if (child_tid < 0){
    process_table[pid].state = PROC_FREE;
    return -1;
  }
  process_table[parent_process].children += 1;
  process_table[pid].state = PROC_RUNNING;
  spinlock_release(&process_lock);
  thread_run(child_tid);
  return pid;
}
\end{lstlisting}

\subsubsection{\texttt{process\_finish}}


\begin{lstlisting}
/* Stop the process and the thread it runs in. Sets the return value as well */
void process_finish(int retval) {
  thread_table_t *thr;
  interrupt_status_t intr_status;
  
  intr_status = _interrupt_disable();
  spinlock_acquire(&process_lock);
  process_id_t current_process = process_get_current_process();
  process_table[current_process].state = PROC_DYING;
  process_table[current_process].retval = retval;
  sleepq_wake_all(&(process_table[current_process]));
  spinlock_release(&process_lock);
  _interrupt_set_state(intr_status);

  thr = thread_get_current_thread_entry();
  vm_destroy_pagetable(thr->pagetable);
  thr->pagetable = NULL;
  thread_finish();
}
\end{lstlisting}

\subsubsection{\texttt{process\_join}}


\begin{lstlisting}
int process_join(process_id_t pid) {
  interrupt_status_t intr_status;
  int retval;

  intr_status = _interrupt_disable();
  spinlock_acquire(&process_lock);

  if (process_get_current_process() != process_table[pid].parentid) return -1;
  while (process_table[pid].state != PROC_DYING) {
    sleepq_add(&(process_table[pid]));
    spinlock_release(&process_lock);
    thread_switch();
    spinlock_acquire(&process_lock);
  }

  retval = process_table[pid].retval;
  spinlock_release(&process_lock);
  _interrupt_set_state(intr_status);

  return retval;
} 
\end{lstlisting}

\subsubsection{\texttt{process\_init}}
Process init simply runs through the process table, setting its state to \texttt{FREE} and initialising all other values its values to defaults as well. It also resets the spinlock, to make sure it has its default state.
\begin{lstlisting}
void process_init() {
  int i;
  spinlock_reset(&process_lock);
  for (i=0; i<CONFIG_MAX_PROCESSES; i++) {
    process_table[i].state        = PROC_FREE;
    process_table[i].id           = -1;
    process_table[i].parentid     = -1;
    process_table[i].name[0]      = '\0'; 
    process_table[i].retval       = -1;
    process_table[i].children     = 0;
  }
}
\end{lstlisting}

\section{System Calls for User-Process Control in Buenos}
We now modify \texttt{syscall.c} to use the functions implemented in \texttt{process.c}. We simply add functions we can call when the appropriate syscalls are invoked (see below).
\begin{lstlisting}
int syscall_exec(char const* filename){
  int pid = process_spawn(filename);
  return pid;
}
void syscall_exit(int retval){
  process_finish(retval);
}
int syscall_join(int pid){
  return process_join(pid);
}
\end{lstlisting}

We now add cases to the syscalls in \texttt{syscall.c}, that loads and writes to registers, calling the functions defined above.
\begin{lstlisting}
switch(user_context->cpu_regs[MIPS_REGISTER_A0])
...
case SYSCALL_EXEC:
    V0 = syscall_exec((char *) A1);
    break;
case SYSCALL_EXIT:
    syscall_exit(A1);
    break;
case SYSCALL_JOIN:
    V0 = syscall_join(A1);
    break;
...
\end{lstlisting}

\end{document}