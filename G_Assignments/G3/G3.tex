\documentclass[a4paper,12pt]{article}

% Standard configuration
\usepackage[english]{babel}
\usepackage[utf8]{inputenc}
\usepackage[T1]{fontenc}
\usepackage{hyperref}
\usepackage{setspace}
\onehalfspacing
\usepackage{upquote}
\usepackage{amsmath,amssymb}
\usepackage{tikz}
\usepackage[margin=1.3in]{geometry}
\usepackage[numbers]{natbib}

% Listings configuration
\usepackage{color}
\definecolor{listinggray}{gray}{0.9}
\usepackage{multirow,bigdelim}
\usepackage{listings}
\lstset{
	language=,
	literate=
		{æ}{{\ae}}1
		{ø}{{\o}}1
		{å}{{\aa}}1
		{Æ}{{\AE}}1
		{Ø}{{\O}}1
		{Å}{{\AA}}1,
	backgroundcolor=\color{listinggray},
	tabsize=3,
	rulecolor=,
	basicstyle=\scriptsize,
	upquote=true,
	aboveskip={1.5\baselineskip},
	columns=fixed,
	showstringspaces=false,
	extendedchars=true,
	breaklines=true,
	prebreak =\raisebox{0ex}[0ex][0ex]{\ensuremath{\hookleftarrow}},
	frame=single,
	showtabs=false,
	showspaces=false,
	showstringspaces=false,
	identifierstyle=\ttfamily,
	keywordstyle=\color[rgb]{0,0,1},
	commentstyle=\color[rgb]{0.133,0.545,0.133},
	stringstyle=\color[rgb]{0.627,0.126,0.941},
}
%captions on listings
\usepackage[center,font=small,labelfont=bf,textfont=it]{caption}

% Header configuration
\usepackage{fancyhdr}
\lhead[]{} %clear standard settings
\chead[]{} %clear standard settings
\rhead[]{} %\rightmark} %current section
\lfoot[]{} %clear standard settings
\cfoot[]{\thepage} %current page number 
\rfoot[]{} %clear standard settings

\title{OSM\\G assignment 3}
\author{Tobias Hallundbæk Petersen (xtv657)\\Ola Rønning (vdl761)\\Nikolaj Høyer (ctl533)}
\date{March 3\textsuperscript{rd}, 2014}

\begin{document}
\maketitle
\tableofcontents
\newpage

\section{Task 1: A Thread-Safe Stack}

\section{Task 2: Userland Semaphores for Buenos}
We have added the following cases to \texttt{proc/syscall.c}:

\lstinputlisting[language=C, firstline=120, lastline=131]{../../proc/syscall.c}

The function calls stated above resides in their own file, an addition to \texttt{proc}, namely \texttt{proc/usr\_semaphore.c} and \texttt{proc/usr\_semaphore.h}. Hence, in \texttt{syscall.c} we add the line \texttt{\#include "proc/usr\_semaphore.h"}. We have also added a new syscall to \texttt{proc/syscall.h}: \texttt{\#define SYSCALL\_SEM\_DESTROY 0x303} to cover the case of this specific syscall.

In \texttt{syscall.h} we define the new user land semaphore structure, which contains of a pointer to a kernel semaphore and a name of the userland semaphore, along with function definitions.

\lstinputlisting[language=C]{../../proc/usr_semaphore.h}

The real functionality is implemented in \texttt{proc/syscall.c}. 

In addition to the functions specified in the assignment text, we have added an init function and added the following lines to \texttt{init/main.c}:

\lstinputlisting[language=C, firstline=213, lastline=214]{../../init/main.c}

The init function simply runs through our userland semaphore table, which we have defined in \texttt{proc/usr\_semaphore.c}, setting all its user semaphores to \texttt{NULL}. This table has a size that is a constant fraction of the kernel semaphores, we have set it to \texttt{CONFIG\_MAX\_SEMAPHORES / 8}. Since the userland semaphores are actually based on kernel semaphores (creating a new userland semaphore will actually create a kernel semaphore), we could get in a situation where the user could fully populate the semaphore table, making the kernel routines unable to create any semaphores at all.

The open function takes care of either retrieving an existing user semaphore (value $< 0$) or creating a new user semaphore (value $>= 0$). Since we're accessing the user semaphore table, we need to disable interrupts and set a spinlock. Most of the functionality revolves around accessing the user semaphore table, comparing the \texttt{name} input variable to the names stored in the \texttt{usr\_sem\_t} structures stored in the semaphore table. The inline comments should describe what happens, and when.

The \texttt{sem\_p} and \texttt{sem\_v} function simply calls the underlying kernel semaphore functionality - since this in itself provides interruption disabling and spinklocks, we don't need to apply that here.

The destroy function simply runs through the user semaphore table, invalidating all the semaphores and freeing the user semaphore table slots.

The code for \texttt{proc/usr\_semaphores.c} is shown below:

\lstinputlisting[language=C]{../../proc/usr_semaphore.c}























\end{document}